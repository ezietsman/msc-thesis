\chapter{SALT RSS raw image preparation scripts}
\label{RSSclean}

These scripts were written to automatically clean all the raw images in batch fashion. See the text for details.
The script \texttt{pysalt.py} calls functions defined in file \texttt{pysaltlib.py}. This file is included 
in section \ref{pysaltlib}.

The scripts were written using the \texttt{Python} language. They need the third-party tools, \texttt{numpy},
\texttt{numarray} and \texttt{pyfits} to be installed.


\section{pysalt.py}
\label{pysalt}

\begin{footnotesize} 
\begin{verbatim}
#!/usr/bin/python
# script to automate some of the tasks that needed to be done manually 
# in IRAF for SALT spectra designed to work on raw SALT data i.e. Files 
# containing 1 header unit and 6 image units

# IMPORTANT: script assumes 2x2 pre-binning since my data was like this. 

# Run in directory containing raw SALT spectra. New FITS files are written 
# with suffix _12cc, _34cc and _56cc for ccd 1 to 3

# New images are bias subtracted using median of bias section, 
# gain corrected using custom gain values for the separate 
# amplifiers and are mosaiced to give the original ccd images. 
# The dead column between amps are interpolated.

# written by Ewald Zietsman 04/05/2007 


import pyfits
import os
import numpy as N
import numarray as num  # still need numarray since pyfits use that. 
import string as S

import pysaltlib as pslib

# PARAMETERS- change these for specific dataset
#------------------------------------------------------------------------#

# gains
# !!!! Change these for specific dataset !!!!
# default values determined from eg21 spectrophotometric standard 
# using imstat on edges of ccd-> pwoudt

gain = [10.5,11.3,10.7,10.0,9.5,9.25]

# change these as desired. Make sure they are in either the 
# Primary or first extension headers

header_entries = ['RA','DEC','OBJECT','UTC-OBS','OBSERVAT','DATE-OBS',
                  'GAIN','RDNOISE','EXPTIME','CTYPE1','CTYPE2','EQUINOX',
                  'CRPIX1','CRPIX2','CRVAL1','CRVAL2','CDELT1','CDELT2',
                  'BITPIX','NAXIS','NAXIS1','NAXIS2','PCOUNT','GCOUNT',
                  'CCDSUM']

#------------------------------------------------------------------------#
# Hopefully no changes below this will be needed

# first find all files in the directory
fitslist = os.listdir('.')

# now open every fits image,split it into its parts and update headers
for fits in fitslist:
    # use try: except: blocks to skip non-FITS files
    try:
        # empty list to keep images in
        images = []
        headers = []

        rootname = fits[:-5]

        # now open the fits file.
        HDU = pyfits.open(fits)
        print '\nProcessing: ',fits
        # read every image and apply gain corrections, 
        # mosaic back into separate ccd images
        for i in range(1,len(HDU)):
            header = HDU[i].header

            # get the bias strip and bias correct the images
            print '\tCalculating bias: '
            biassec = pslib.returnbiassec(header['biassec'][1:-1])
            data = HDU[i].data
            data = pslib.subtract_bias(data,biassec)

            # multiply images with gain corrections specified 
            # at beginning of file
            print '\tApplying gain correction: ', gain[i-1]
            data *= gain[i-1]

            # save bias corrected images in list
            images.append(data)
            headers.append(header)  

            print ''
            #name = root + '_' + str(i) + '.fits'
            #pyfits.writeto(name,data,header=header)

       # read headers and write to big header file
       # do this properly later
        print "\tReconstructing header : %s " % fits
        header0 = HDU[0].header
        header  = headers[0]
        new_head = pslib.recon_header(header0,header,header_entries)

        # now mosaic arrays to make the full ccd images again and write
        # to disk also muliply 'fudge factor' to make the spectrum 
        # continuous across ccd's
        # !!!!!!!!!  NB convert images to numarray arrays. 
        # Change when pyfits moves to numpy!!!!!!!!!!!!

        print '\tReconstructing CCD images: CCD1'
        frame12 = num.array(pslib.recon_ccd(images[0],images[1],1))
        filename = rootname + '_12cc.fits'
        pyfits.writeto(filename,frame12,header=new_head)

        print '\tReconstructing CCD images: CCD2'
        frame34 = 1.1*num.array(pslib.recon_ccd(images[2],images[3],2))
        filename = rootname + '_34cc.fits'
        pyfits.writeto(filename,frame34,header=new_head)

        print '\tReconstructing CCD images: CCD3'
        frame56 = 1.15*num.array(pslib.recon_ccd(images[4],images[5],3))
        filename = rootname + '_56cc.fits'
        pyfits.writeto(filename,frame56,header=new_head)

    # tried to open non-FITS file
    except:
        # skip this one
        print 'Error: %s not raw SALT spectrum. Skipping.' % fits
        continue
\end{verbatim}

\end{footnotesize}



\section{pysaltlib.py}
\label{pysaltlib}


\begin{footnotesize}
\begin{verbatim}
 
# module that contains useful functions
import string as S
import numpy as N
import pylab as pl


def returnbiassec(bias):
    # function takes the 'biassec' header entry and returns the bias strip 
    # limits xlo,xhi,ylo,yhi values as integers
    temp = S.split(S.strip(bias),',')
    xlo = int(S.split(temp[0],':')[0])
    xhi = int(S.split(temp[0],':')[1])
    ylo = int(S.split(temp[1],':')[0])
    yhi = int(S.split(temp[1],':')[1])
    
    return xlo,xhi,ylo,yhi
    

def subtract_bias(image,biaslimits):
    # takes data section as returned by pyfits and calculates the bias 
    # using median of the biassec as in header entry
    
    xlo,xhi,ylo,yi = biaslimits
    image = N.array(image)
    print '\tBias limits: ',xlo,xhi,ylo,yi
    
    # calculate the median of the bias section
    # flatten changes the 2D array to single column array
    bias = N.median(image[:,xlo:xhi].flatten())
    
    print '\tBias: ', bias
    # subtract bias
    image[:,:] -= bias
    
    return image


def recon_ccd(image1,image2,ccdno):
    # takes images from 2 amps from ccd and reconstructs the
    # original ccd image
    # also correct the dead column between the amps
    # ccdno must be 1 to correct the second dead line on ccd1 on the RSS

    # !!!!! N.B. These assume 2x2 prebinning.
    # Take care if binning is different !!!!!
    temp = N.zeros(205824)
    temp.shape = (201,1024)
    temp[:,0:512] = image1[:,25:537]
    temp[:,512:1024] = image2[:,25:537]

    # correct the dead column
    for i in range(201):
        temp[i,512] = (temp[i,511] + temp[i,513]) / 2.0

    # correct the funny column on RSS's ccd1 (col 465 or so)
    if ccdno == 1:
        for i in range(201):
            temp[i,463] = (temp[i,462] + temp[i,465]) / 2.0
            temp[i,464] = (temp[i,462] + temp[i,465]) / 2.0
    return temp


def recon_header(header0,header,entries):
    #function attempts to reconstruct header for output FITS file.
    new_head = header

    # get entries frim Primary HDU and construct new header.
    for entry in entries:
        # try to get ebtry from header0
        try:
            new_head.update(entry,header0[entry])
        # maybe its in the secondary headers...
        except:
            new_head.update(entry,header[entry])

    # put correct DATASEC value in FITS header
    new_head.update('DATASEC','[1:2024,1:201]')

    return new_head

\end{verbatim}
 

\end{footnotesize}



\chapter{Spectrum extraction script}

\label{spec_extr}

\begin{footnotesize}
 


\begin{verbatim}
 
# pyraf script to do spectrum extractions

import os
import pylab as pl
import string
import pyfits as pf

# start IRAF
cd = os.getcwd()
try:
    print 'Trying /home/lemoen/'
    os.chdir('/home/lemoen/')
except:
    print 'We must be on corvus then... trying /home/ezietsman/'
    os.chdir('/home/ezietsman/')
    
from pyraf import iraf
os.chdir(cd)
# clean up previous reductions
os.system('rm -v e*')
os.system('rm -v arcA*')
os.system('rm -v arcB*')

# load NOAO package
iraf.noao()
iraf.astutil()
iraf.onedspec()
iraf.twodspec()
iraf.apextract()
iraf.apextract(dispaxis='1')
# extract the object spectra on each of the ccds


for ccd in ['_12cc','_34cc','_56cc']:
    for i in range(61,396):
        print '\n\nExtracting object spectrum number  %s on ccd %s \n\n\n ' % (i,ccd[1:3])
        
        # extract object spectra
        iraf.apall(input='P20060817%04d%s' % (i,ccd),\
        output='eP20060817%04d%s' % (i,ccd),\
        format='onedspec',\
        interactive='no',\
        find='no',\
        recenter='yes',\
        resize='no',\
        edit='no', \
        nfind='1',\
        references='P200608170061%s' % ccd, \
        trace='yes',\
        fittrace='no',\
        extract='yes',\
        extras='no',\
        review='no',\
        t_nsum='25',\
        t_step='25',\
        background='fit',\
        clean='yes',\
        weights='variance',\
        pfit='fit1d',\
        gain='1.0',\
        readnoise='10.2')
        
        # extract arc spectra
        iraf.apall(input='P200608170060%s' % ccd,\
        output='arcA%04d%s' % (i,ccd),\
        format='onedspec',\
        interactive='no',\
        find='no',\
        recenter='no',\
        resize='no',\
        edit='no', \
        nfind='1',\
        references='P20060817%04d%s' % (i,ccd), \
        trace='no',\
        fittrace='no',\
        extract='yes',\
        extras='no',\
        review='no',\
        t_nsum='25',\
        t_step='25',\
        background='none',\
        clean='yes',\
        weights='variance',\
        pfit='fit1d',\
        gain='1.0',\
        readnoise='10.2')
        
        iraf.apall(input='P200608170396%s' % ccd,\
        output='arcB%04d%s' % (i,ccd),\
        format='onedspec',\
        interactive='no',\
        find='no',\
        recenter='no',\
        resize='no',\
        edit='no', \
        nfind='1',\
        references='P20060817%04d%s' % (i,ccd), \
        trace='no',\
        fittrace='no',\
        extract='yes',\
        extras='no',\
        review='no',\
        t_nsum='25',\
        t_step='25',\
        background='none',\
        clean='yes',\
        weights='variance',\
        pfit='fit1d',\
        gain='1.0',\
        readnoise='10.2')
        
        
        # add the reference spectra keywords to the header
        
        iraf.hedit(images='eP20060817%04d%s.0001' % (i,ccd) ,\
        fields='REFSPEC1',\
        value='arcA%04d%s.0001 0.5' % (i,ccd),\
        add='yes',\
        verify='no',\
        show='yes',\
        update='yes')
        
        iraf.hedit(images='eP20060817%04d%s.0001' % (i,ccd) ,\
        fields='REFSPEC2',\
        value='arcB%04d%s.0001 0.5' % (i,ccd),\
        add='yes',\
        verify='no',\
        show='yes',\
        update='yes')
        
        # run setjd on each output spectrum
        iraf.setjd(images='eP20060817%04d%s.0001' % (i,ccd),\
        date='DATE-OBS',\
        time='UTC-OBS',\
        exposure='EXPTIME',\
        observatory='saao',\
        ra='RA',\
        dec='DEC',\
        epoch='EQUINOX',\
        jd='JD',\
        hjd='HJD',\
        ljd='LJD',\
        utdate='yes',\
        uttime='yes',\
        listonly='no')
        
        # add sidereal time to header
        head = pf.getheader('eP20060817%04d%s.0001.fits' % (i,ccd))
        ymd = string.split(head['date-obs'],sep='-')
        year = int(ymd[0])
        month = int(ymd[1])
        day = int(ymd[2])
        ut = string.split(head['utc-obs'],sep=':')
        zt = string.join([str(int(ut[0])+2),ut[1],ut[2]],':')
        temp = iraf.asttimes(observatory='saao',header='no',year=year,\
        month=month, day=day, time=zt,Stdout=1)
        st = string.split(temp[0])[8]
        head.update('st',st)
        data = pf.getdata('eP20060817%04d%s.0001.fits' % (i,ccd))
        pf.writeto('eP20060817%04d%s.0001.fits' % (i,ccd),data,header=head,clobber=True)
        
        
        # calculate airmass and add it to header
        
        iraf.setairmass(images='eP20060817%04d%s.0001'%(i,ccd),\
        outtype='middle',\
        observatory='saao',\
        ut='utc-obs',\
        equinox='equinox')






\end{verbatim}

\end{footnotesize} 
